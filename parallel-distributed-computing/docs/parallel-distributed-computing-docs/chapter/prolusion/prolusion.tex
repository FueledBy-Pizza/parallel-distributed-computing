\begin{chapter}{Prolusion}
	\begin{section}{Goal}
		\par This document offers a comprehensive overview of a project developed in \texttt{C}, consisting of multiple modules delivered in incremental phases.
		\par It is not intended as a user guide, but rather aims to describe the project’s exploration of parallel computing techniques, leveraging \glsxtrlong{hpc} in certain instances.
	\end{section}
	\begin{section}{Environment}
		\par The project was entirely developed on a macOS system with the help of Xcode \glsxtrshort{ide}.\\
		Naturally, this will mainly impact the build process.
	\end{section}
	\clearpage
	\begin{section}{Project directory layout}
		\par The structure of the project’s root directory is outlined below.
		\medskip
		\dirtree{%
			.1 parallel-distributed-computing/.
			.2 common/.
			.2 hpc/.
			.3 gemm/.
			.3 matmatblock/.
			.3 matmatdist/.
			.3 matmatthread/.
			.2 laplace/.
			.2 maxsum/.
			.2 ringsum/.
			.2 parallel-distributed-computing.entitlements.
		}
		\medskip
		\texttt{common} package serves as a library of utility functions designed to support and be reused by various modules across the project.
		\vspace{\baselineskip}
		\par The remaining directories represent the individual project modules, which constitute the deliverables of the project. Within each module, the directory structure follows a standard format:
		\medskip
		\dirtree{%
			.1 <module>/.
			.2 build/.
			.3 deploy-cluster.pbs.
			.3 Makefile.
			.2 src/.
			.3 <module>/.
			.4 <module>.c.
			.4 <module>.h.
			.3 main.c.
			.2 config.sh.
			.2 run.sh.
		}
		\medskip
		\par Most parts of the \texttt{main.c} files are provided by the project supervisor.
	\end{section}
	\clearpage
	\begin{section}{Build}
		\par The project was primarily compiled using the Clang compiler.\\
		The build process was carried out either through the \texttt{Makefile} (some of which support compilers other than Clang) or via Xcode.
		\par Regardless of the build process, every module of the project was compiled with:
		\begin{itemize}
			\item \texttt{-O3} optimization flag to maximize performance.
		\end{itemize}
		\begin{subsection}{Libraries}
			The following are the dynamically linked libraries integrated into the project.
			\begin{itemize}
				\item \texttt{math.h}
				\item \texttt{mpi.h}
				\item \texttt{omp.h}
				\item \texttt{stdio.h}
				\item \texttt{stdbool.h}
				\item \texttt{stdlib.h}
				\item \texttt{sys/time.h}
				\item \texttt{unistd.h}
			\end{itemize}
		\end{subsection}
		\begin{subsection}{Xcode}
			\par When it came to build with the Xcode, the development process adhered to the workflow and conventions defined by the chosen \glsxtrshort{ide}, leveraging its built-in tools and features to organize and manage the project. Particularly, this includes:
			\begin{itemize}
				\item Xcode targets
				\item Xcode schemes
				\item Xcode \texttt{.entitlements} file
			\end{itemize}
		\end{subsection}
	\end{section}
	\clearpage
	\begin{section}{Run}
		\begin{subsection}{Network Requirement}
			\par Running an \glsxtrshort{mpi} module with no internet connection, makes the following error occur:
			\begin{verbatim}
					[Giulianos-MacBook-Pro.local:05355] ptl_tool: problems getting address for
					index 0 (kernel index -1)
					--------------------------------------------------------------------------
					The PMIx server's listener thread failed to start. We cannot continue.
					--------------------------------------------------------------------------
					Program ended with exit code: 213
			\end{verbatim}
		\end{subsection}
	\end{section}
\end{chapter}