\begin{chapter}{Prolusion}
	\begin{section}{Goal}
		\par This document provides a comprehensive overview of a project developed in \texttt{C}, consisting of multiple modules delivered in incremental phases. The purpose of the project is to experiment parallel computing techniques and occasionally exploiting \glsxtrlong{hpc}.
	\end{section}
	\begin{section}{Environment}
		\par The project was entirely developed on a macOS system with the help of Xcode \glsxtrshort{ide}. Of course, this will mainly impact the build process.\gls{bug}
	\end{section}
	\begin{section}{Project directory layout}
		\par The structure of the project’s root directory is outlined below.
		\medskip
		\dirtree{%
			.1 parallel-distributed-computing/.
			.2 common/.
			.2 hpc/.
			.3 gemm/.
			.3 matmatblock/.
			.3 matmatdist/.
			.3 matmatthread/.
			.2 laplace/.
			.2 maxsum/.
			.2 ringsum/.
			.2 parallel-distributed-computing.entitlements.
		}
		\medskip
		\texttt{common} package serves as a library of utility functions designed to support and be reused by various modules across the project.
		\clearpage
		\par The remaining directories represent the individual project modules, which constitute the deliverables of the project. Under each module the structure is a standard one:
		\medskip
		\dirtree{%
			.1 <module>/.
			.2 build/.
			.3 deploy-cluster.pbs.
			.3 Makefile[.gcc].
			.2 src/.
			.3 <module>/.
			.3 main.c.
			.2 config.sh.
			.2 run.sh.
		}
		\medskip
		\par Most parts of the \texttt{main.c} files are provided by the project supervisor.
	\end{section}
	\begin{section}{Build}
		\par The main compiler of this project is Clang.
		\par
		\begin{subsection}{Xcode}
			\par
		\end{subsection}
	\end{section}
    \begin{section}{Figures}
    	\begin{subsection}{Option inkscapelatex}
		\par This is an imported SVG with \LaTeX embedded text.
		\begin{figure}[ht]
			\centering
			\includesvg[inkscapelatex=true]{drawing.svg}
			\caption{\emph{True inkscapelatex option}}
			\label{fig:latexembedded}
		\end{figure}
		\par This is an imported SVG without \LaTeX embedded text.
		\begin{figure}[ht]
			\centering
			\includesvg[inkscapelatex=false]{drawing.svg}
			\caption{\emph{False inkscapelatex option}}
			\label{fig:latexnotembedded}
		\end{figure}
	\end{subsection}
    \end{section}
\end{chapter}