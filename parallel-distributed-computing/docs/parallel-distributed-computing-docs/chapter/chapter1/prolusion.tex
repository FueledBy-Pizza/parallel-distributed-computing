\begin{chapter}{Prolusion}
	\begin{section}{Goal}
		\par This document offers a comprehensive overview of a project developed in \texttt{C}, consisting of multiple modules delivered in incremental phases.
		\par It is not intended as a user guide, but rather aims to describe the project’s exploration of parallel computing techniques, leveraging \glsxtrlong{hpc} in certain instances.
	\end{section}
	\begin{section}{Environment}
		\par The project was entirely developed on a macOS system with the help of Xcode \glsxtrshort{ide}.\\
		Of course, this will mainly impact the build process.
	\end{section}
	\clearpage
	\begin{section}{Project directory layout}
		\par The structure of the project’s root directory is outlined below.
		\medskip
		\dirtree{%
			.1 parallel-distributed-computing/.
			.2 common/.
			.2 hpc/.
			.3 gemm/.
			.3 matmatblock/.
			.3 matmatdist/.
			.3 matmatthread/.
			.2 laplace/.
			.2 maxsum/.
			.2 ringsum/.
			.2 parallel-distributed-computing.entitlements.
		}
		\medskip
		\texttt{common} package serves as a library of utility functions designed to support and be reused by various modules across the project.
		\vspace{\stretch{1}}
		\par The remaining directories represent the individual project modules, which constitute the deliverables of the project. Under each module the structure is a standard one:
		\medskip
		\dirtree{%
			.1 <module>/.
			.2 build/.
			.3 deploy-cluster.pbs.
			.3 Makefile[.gcc].
			.2 src/.
			.3 <module>/.
			.3 main.c.
			.2 config.sh.
			.2 run.sh.
		}
		\medskip
		\par Most parts of the \texttt{main.c} files are provided by the project supervisor.
	\end{section}
	\clearpage
	\begin{section}{Build}
		\par The project was primarily compiled using the Clang compiler.\\
		The build process was carried out either through the \texttt{Makefile} (some of which supports compilers other than Clang) or via Xcode. In any case, the project was compiled with the \texttt{-O3} optimization flag to maximize performance.
		\begin{subsection}{Xcode}
			\par
		\end{subsection}
		\begin{subsection}{Libraries}
			The following are the dynamically linked libraries integrated into the project.
			\begin{itemize}
				\item \texttt{math.h}
				\item \texttt{mpi.h}
				\item \texttt{omp.h}
				\item \texttt{stdio.h}
				\item \texttt{stdbool.h}
				\item \texttt{stdlib.h}
				\item \texttt{sys/time.h}
				\item \texttt{unistd.h}
			\end{itemize}
		\end{subsection}
		\begin{subsection}{Run}
		\end{subsection}
	\end{section}
    \begin{section}{Figures}
    	\begin{subsection}{Option inkscapelatex}
		\par This is an imported SVG with \LaTeX embedded text.
		\begin{figure}[ht]
			\centering
			\includesvg[inkscapelatex=true]{drawing.svg}
			\caption{\emph{True inkscapelatex option}}
			\label{fig:latexembedded}
		\end{figure}
		\par This is an imported SVG without \LaTeX embedded text.
		\begin{figure}[ht]
			\centering
			\includesvg[inkscapelatex=false]{drawing.svg}
			\caption{\emph{False inkscapelatex option}}
			\label{fig:latexnotembedded}
		\end{figure}
	\end{subsection}
    \end{section}
\end{chapter}