\begin{chapter}{MatMatikj}
    \begin{section}{Problem at a glance}
        \par Given matrices $A$, $B$, $C$, the goal is to compute the \glsxtrshort{gemm} operation $C = C + AB$ by iterating over the matrices while leveraging the ikj ordering of indices. This increase cache memory utilization.
    \end{section}
    \begin{section}{Algorithm}
        \begin{subsection}{Parallelization}
            No parallelization is planned for this algorithm.
        \end{subsection}
        \begin{subsection}{Pseudocode}
            \textbf{Input}:
            \begin{itemize}
                \item $A^{N1 \times N2}$, $B^{N2 \times N3}$, $C^{N1 \times N3}$ matrices
            \end{itemize}
            \textbf{Steps}:
            \begin{itemize}
                \item Iterate over the three input matrices using the `ikj' order
                \item Compute $C = C + AB$ incrementally for each step
            \end{itemize}
            \input{\pseudocodepath matmatikj}
        \end{subsection}
    \end{section}
    If a matrix row fits entirely in the cache line as a cache block, number of main memory accesses is $\mathcal{O}(N^2)$, otherwise it $\mathcal{O}(N^3)$.
\end{chapter}