\begin{chapter}{MatMatThread}
    \begin{section}{Problem at a glance}
        \par Given matrices $A$, $B$, $C$, $blockN1$, $blockN2$, $blockN3$, a number of $NTROW \times NTCOL$ threads, the goal is to compute the \glsxtrshort{gemm} operation $C = C + AB$ in a multithreaded MatMatBlock approach.
    \end{section}
    \begin{section}{Algorithm}
        \begin{subsection}{Parallelization}
            \par The algorithm employs a multithreaded approach where each thread is responsible for executing a single \glsxtrshort{gemm} operation of a $C$ block.
            \par Each thread can compute independently its local \glsxtrshort{gemm} operation, no synchronization is needed among the threads.
        \end{subsection}
        \begin{subsection}{Pseudocode}
            \textbf{Input}:
            \begin{itemize}
                \item $A^{N1 \times N2}$, $B^{N2 \times N3}$, $C^{N1 \times N3}$ matrices
                \item $blockN1$, $blockN2$, $blockN3$ block dimensions of $N1$, $N2$, $N3$ (used by MatMatBlock algorithm)
                \item $NTROW, NTCOL$ number of thread(s) on rows and columns of $C$ that will be assigned to compute part of the whole \glsxtrshort{gemm} operation on $C$
            \end{itemize}
            \textbf{Steps}:
            \begin{itemize}
                \item Compute the indices of the global matrix corresponding to the local matrix of the thread.
                \item Execute MatMatBlock algorithm
            \end{itemize}
            \input{\pseudocodepath matmatthread}
        \end{subsection}
        \begin{figure}[ht]
            \centering
            \includesvg[inkscapelatex=false]{matmatthread.svg}
            \label{fig:matmatthread}
        \end{figure}
        \clearpage
        \begin{subsection}{Performance Analysis}
            \begin{figure}[ht]
                \centering
                \includesvg[inkscapelatex=false, width=\textwidth]{matmatthread-exectime.svg}
                \label{fig:matmatthread-exectime}
            \end{figure}
            \begin{figure}[ht]
                \centering
                \includesvg[inkscapelatex=false, width=\textwidth]{matmatthread-speedup.svg}
                \label{fig:matmatthread-speedup}
            \end{figure}
            \begin{figure}[ht]
                \centering
                \includesvg[inkscapelatex=false, width=\textwidth]{matmatthread-efficiency.svg}
                \label{fig:matmatthread-efficiency}
            \end{figure}
        \end{subsection}
    \end{section}
\end{chapter}