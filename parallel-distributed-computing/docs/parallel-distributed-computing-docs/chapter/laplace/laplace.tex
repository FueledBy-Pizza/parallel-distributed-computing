\begin{chapter}{Laplace}
    \begin{section}{Problem at a glance}
        \par Given a matrix and a number of iterations, the objective is to compute the Laplacian over the specified iterations.
    \end{section}
    \begin{section}{Algorithm}
        \begin{subsection}{Parallelization}
            \par The algorithm employs a multiprocess approach. A synchronization between processes is needed from the outset.
            \par The global matrix is distributed among the processes, harnessing the advantages of parallelism, representing a spatial advantage other than a temporal one.
        \end{subsection}
        \begin{subsection}{Pseudocode}
            \textbf{Input}:
            \begin{itemize}
                \item local process matrix
                \item number of iteration(s)
            \end{itemize}
            \textbf{Steps}:\\
            For each iteration:
            \begin{enumerate}
                \item If $P_i \neq P_1$ Then
                \begin{enumerate}
                    \item Send to $P_{i-1}$ the local first row
                    \item Receive from $P_{i-1}$ its last row
                \end{enumerate}
                \item If $P_i \neq P_n$ Then
                \begin{enumerate}
                    \item Receive from $P_{i+1}$ its first row
                    \item Send to $P_{i+1}$ the local last row
                \end{enumerate}
                \item Compute the Laplacian considering only the local inner matrix, the matrix boundaries are left out.
                \item If $P_i \neq P_1$
                \begin{enumerate}
                    \item Compute the Laplacian considering only the local top row, with the auxiliary last row from the previous process.
                \end{enumerate}
                \item If $P_i \neq P_n$
                \begin{enumerate}
                    \item Compute the Laplacian considering only the local bottom row, with the auxiliary first row from the next process.
                \end{enumerate}
            \end{enumerate}
            An auxiliary matrix is involved to facilitate the computation.
            \input{\pseudocodepath laplace}
        \end{subsection}
    \end{section}
\end{chapter}