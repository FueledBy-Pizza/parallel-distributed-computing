\begin{chapter}{Laplace}
    \begin{section}{Problem at a glance}
        \par Given a matrix and a number of iterations, the objective is to compute the Laplacian over the specified iterations.
    \end{section}
    \begin{section}{Algorithm}
        \begin{subsection}{Parallelization}
            \par The algorithm employs a multiprocess approach. A synchronization between processes is needed from the outset.
            \par The input matrix is distributed among the processes, harnessing the advantages of parallelism, representing a spatial advantage other than a temporal one.
        \end{subsection}
        \begin{subsection}{Pseudocode}
            Input:
            \begin{itemize}
                \item matrix
                \item number of iteration(s)
            \end{itemize}
            Iteration step(s):
            \begin{enumerate}
                \item Assign rows to processes.
                \item For each process $P_i$
                \begin{enumerate}
                    \begin{item}
                        If $P_i \neq P_1$
                        \begin{itemize}
                            \item Send to $P_{i-1}$ the local first row
                            \item Receive from $P_{i-1}$ its last row
                        \end{itemize}
                    \end{item}
                    \begin{item}
                        If $P_i \neq P_n$
                        \begin{itemize}
                            \item Receive from $P_{i+1}$ its first row
                            \item Send to $P_{i+1}$ the local last row
                        \end{itemize}
                    \end{item}
                    \item Compute the Laplacian considering only the local inner matrix, the matrix boundaries are left out.
                    \begin{item}
                        If $P_i \neq P_1$
                        \begin{itemize}
                            \item Compute the Laplacian considering only the local top row, with the auxiliary last row from the previous process.
                        \end{itemize}
                    \end{item}
                    \begin{item}
                        If $P_i \neq P_n$
                        \begin{itemize}
                            \item Compute the Laplacian considering only the local bottom row, with the auxiliary first row from the next process.
                        \end{itemize}
                    \end{item}
                \end{enumerate}
            \end{enumerate}
            An auxiliary matrix is used to facilitate the computation.
            \input{\pseudocodepath laplace}
        \end{subsection}
    \end{section}
\end{chapter}