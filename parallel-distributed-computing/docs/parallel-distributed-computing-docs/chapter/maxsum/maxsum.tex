\begin{chapter}{Maxsum}
    \begin{section}{Problem at a glance}
        \par Given a matrix, the objective is to identify the row with the maximum squared root sum.
    \end{section}
    \begin{section}{Algorithm}
        \begin{subsection}{Parallelization}
            \par The algorithm employs a multithreaded approach, each thread can compute indipendently its local maximum squared root sum. Once a thread has completed its task, a synchronization is needed with the other threads in order to determine the overall maximum squared root sum.
        \end{subsection}
        \begin{subsection}{Pseudocode}
            \textbf{Input}:
            \begin{itemize}
                \item matrix
                \item number of thread(s) for parallel execution
            \end{itemize}
            \textbf{Steps}:
            \begin{enumerate}
                \item Assign rows to threads.
                \item For each thread:
                \begin{enumerate}
                    \item Compute the local maximum squared root sum.
                    \item Perform a thread-safe update of the overall maximum squared root sum variable.
                \end{enumerate}
                \item Return the overall maximum squared root sum.
            \end{enumerate}
            \input{\pseudocodepath maxsum}
        \end{subsection}
        \clearpage
        \begin{subsection}{Performance Analysis}
            \begin{figure}[ht]
                \centering
                \includesvg[inkscapelatex=false, width=\textwidth]{maxsum-exectime.svg}
                %\caption{\emph{The neuron structure}}
                \label{fig:maxsum-exectime}
            \end{figure}
            \begin{figure}[ht]
                \centering
                \includesvg[inkscapelatex=false, width=\textwidth]{maxsum-speedup.svg}
                \label{fig:maxsum-speedup}
            \end{figure}
            \begin{figure}[ht]
                \centering
                \includesvg[inkscapelatex=false, width=\textwidth]{maxsum-efficiency.svg}
                \label{fig:maxsum-efficiency}
            \end{figure}
        \end{subsection}
    \end{section}
\end{chapter}